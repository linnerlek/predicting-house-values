%Two resources useful for abstract writing.
% Guidance of how to write an abstract/summary provided by Nature: https://cbs.umn.edu/sites/cbs.umn.edu/files/public/downloads/Annotated_Nature_abstract.pdf %https://writingcenter.gmu.edu/guides/writing-an-abstract
\chapter*{\center \Large  Abstract}

Predicting how home values change each year is important for investors, planners, and anyone involved in housing decisions. This project focuses on forecasting year-over-year (YoY) home price growth at the ZIP-code level in Georgia. We used Zillow Home Value Index (ZHVI) data along with socioeconomic and demographic information from the American Community Survey (ACS) to build a dataset that captures both housing trends and local conditions.

Each row in our dataset represents a ZIP code in a specific year, with features like volatility, average monthly growth, and previous YoY changes from the ZHVI time series. We also added county-level ACS variables like education levels, poverty rates, unemployment, and household structure to capture broader context.

To find the most useful features, we used two different selection methods: mutual information and f-regression. Then we trained three models — Decision Tree, K-Nearest Neighbors (KNN), and Random Forest — tuning them using GridSearchCV. The models were evaluated using R², RMSE, MAE, and SMAPE on a ZIP-based test set. Random Forest performed the best overall, while the other models showed signs of overfitting unless properly tuned.

Overall, we found that recent YoY trends and economic indicators were the most reliable signals for short-term price growth. This setup could be extended in the future to make longer-term predictions as more recent ACS data becomes available.

~\\[1cm]
\noindent\textbf{Keywords:} housing prediction, machine learning, ZHVI, ZIP code, ACS

\vfill
\noindent
\textbf{Report's total word count:} 3637 words





