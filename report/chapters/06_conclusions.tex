\chapter{Conclusions and Future Work}
\label{ch:con}

\section{Conclusions}

This project set out to predict one-year-ahead home value growth at the ZIP code level in Georgia using housing price trends from Zillow and demographic context from ACS data. The goal was to build a model that could pick up on both short-term market signals and longer-term economic patterns.

We built a structured pipeline that cleaned and combined both datasets, engineered useful features, and tested three models. Random Forest gave the best results overall, handling complex and noisy data without overfitting. It consistently outperformed Decision Tree and KNN, especially on unseen ZIP codes and in years with volatile growth patterns.

The features that mattered most were mostly tied to recent housing trends — like prior YoY growth, volatility, and monthly change — but demographic indicators like poverty and education levels also added value. Our feature selection process helped highlight which signals were reliable, and our ZIP-based test split ensured that models weren’t just memorizing locations.

In the end, we were able to show that even with some limitations, combining housing and economic data can give solid short-term predictions at a local level. This kind of setup could be helpful for anyone trying to understand where growth might happen next — whether that’s a buyer, planner, or researcher.

\section{Future Work}

There are a few clear next steps that could build on this work:

\begin{enumerate}
    \item \textbf{Add material price data.} Including average construction or renovation costs might help explain changes in home value beyond demand. It could give better insight into supply-side effects.

    \item \textbf{Look at longer-term growth.} Instead of just year-over-year changes, forecasting growth over 3 to 5 years would help with planning and might reduce noise in the predictions.

    \item \textbf{Expand to more regions and time periods.} Right now the model is limited to Georgia. Applying it to other states or using older data could test how generalizable it is and reveal regional differences.

    \item \textbf{Build an interactive dashboard.} A simple tool that lets users explore ZIP-level predictions, filter by region or income level, and compare trends could make this more useful in practice.
\end{enumerate}

This project focused on modeling what was possible using the data available. With better geographic granularity and richer features, future versions could become more accurate, more flexible, and easier to use.
